% Options for packages loaded elsewhere
\PassOptionsToPackage{unicode}{hyperref}
\PassOptionsToPackage{hyphens}{url}
%
\documentclass[
]{article}
\usepackage{amsmath,amssymb}
\usepackage{lmodern}
\usepackage{ifxetex,ifluatex}
\ifnum 0\ifxetex 1\fi\ifluatex 1\fi=0 % if pdftex
  \usepackage[T1]{fontenc}
  \usepackage[utf8]{inputenc}
  \usepackage{textcomp} % provide euro and other symbols
\else % if luatex or xetex
  \usepackage{unicode-math}
  \defaultfontfeatures{Scale=MatchLowercase}
  \defaultfontfeatures[\rmfamily]{Ligatures=TeX,Scale=1}
\fi
% Use upquote if available, for straight quotes in verbatim environments
\IfFileExists{upquote.sty}{\usepackage{upquote}}{}
\IfFileExists{microtype.sty}{% use microtype if available
  \usepackage[]{microtype}
  \UseMicrotypeSet[protrusion]{basicmath} % disable protrusion for tt fonts
}{}
\makeatletter
\@ifundefined{KOMAClassName}{% if non-KOMA class
  \IfFileExists{parskip.sty}{%
    \usepackage{parskip}
  }{% else
    \setlength{\parindent}{0pt}
    \setlength{\parskip}{6pt plus 2pt minus 1pt}}
}{% if KOMA class
  \KOMAoptions{parskip=half}}
\makeatother
\usepackage{xcolor}
\IfFileExists{xurl.sty}{\usepackage{xurl}}{} % add URL line breaks if available
\IfFileExists{bookmark.sty}{\usepackage{bookmark}}{\usepackage{hyperref}}
\hypersetup{
  pdftitle={Approximation of the Binomial probabilities in R},
  pdfauthor={Ivan Gercev},
  hidelinks,
  pdfcreator={LaTeX via pandoc}}
\urlstyle{same} % disable monospaced font for URLs
\usepackage[margin=1in]{geometry}
\usepackage{color}
\usepackage{fancyvrb}
\newcommand{\VerbBar}{|}
\newcommand{\VERB}{\Verb[commandchars=\\\{\}]}
\DefineVerbatimEnvironment{Highlighting}{Verbatim}{commandchars=\\\{\}}
% Add ',fontsize=\small' for more characters per line
\usepackage{framed}
\definecolor{shadecolor}{RGB}{248,248,248}
\newenvironment{Shaded}{\begin{snugshade}}{\end{snugshade}}
\newcommand{\AlertTok}[1]{\textcolor[rgb]{0.94,0.16,0.16}{#1}}
\newcommand{\AnnotationTok}[1]{\textcolor[rgb]{0.56,0.35,0.01}{\textbf{\textit{#1}}}}
\newcommand{\AttributeTok}[1]{\textcolor[rgb]{0.77,0.63,0.00}{#1}}
\newcommand{\BaseNTok}[1]{\textcolor[rgb]{0.00,0.00,0.81}{#1}}
\newcommand{\BuiltInTok}[1]{#1}
\newcommand{\CharTok}[1]{\textcolor[rgb]{0.31,0.60,0.02}{#1}}
\newcommand{\CommentTok}[1]{\textcolor[rgb]{0.56,0.35,0.01}{\textit{#1}}}
\newcommand{\CommentVarTok}[1]{\textcolor[rgb]{0.56,0.35,0.01}{\textbf{\textit{#1}}}}
\newcommand{\ConstantTok}[1]{\textcolor[rgb]{0.00,0.00,0.00}{#1}}
\newcommand{\ControlFlowTok}[1]{\textcolor[rgb]{0.13,0.29,0.53}{\textbf{#1}}}
\newcommand{\DataTypeTok}[1]{\textcolor[rgb]{0.13,0.29,0.53}{#1}}
\newcommand{\DecValTok}[1]{\textcolor[rgb]{0.00,0.00,0.81}{#1}}
\newcommand{\DocumentationTok}[1]{\textcolor[rgb]{0.56,0.35,0.01}{\textbf{\textit{#1}}}}
\newcommand{\ErrorTok}[1]{\textcolor[rgb]{0.64,0.00,0.00}{\textbf{#1}}}
\newcommand{\ExtensionTok}[1]{#1}
\newcommand{\FloatTok}[1]{\textcolor[rgb]{0.00,0.00,0.81}{#1}}
\newcommand{\FunctionTok}[1]{\textcolor[rgb]{0.00,0.00,0.00}{#1}}
\newcommand{\ImportTok}[1]{#1}
\newcommand{\InformationTok}[1]{\textcolor[rgb]{0.56,0.35,0.01}{\textbf{\textit{#1}}}}
\newcommand{\KeywordTok}[1]{\textcolor[rgb]{0.13,0.29,0.53}{\textbf{#1}}}
\newcommand{\NormalTok}[1]{#1}
\newcommand{\OperatorTok}[1]{\textcolor[rgb]{0.81,0.36,0.00}{\textbf{#1}}}
\newcommand{\OtherTok}[1]{\textcolor[rgb]{0.56,0.35,0.01}{#1}}
\newcommand{\PreprocessorTok}[1]{\textcolor[rgb]{0.56,0.35,0.01}{\textit{#1}}}
\newcommand{\RegionMarkerTok}[1]{#1}
\newcommand{\SpecialCharTok}[1]{\textcolor[rgb]{0.00,0.00,0.00}{#1}}
\newcommand{\SpecialStringTok}[1]{\textcolor[rgb]{0.31,0.60,0.02}{#1}}
\newcommand{\StringTok}[1]{\textcolor[rgb]{0.31,0.60,0.02}{#1}}
\newcommand{\VariableTok}[1]{\textcolor[rgb]{0.00,0.00,0.00}{#1}}
\newcommand{\VerbatimStringTok}[1]{\textcolor[rgb]{0.31,0.60,0.02}{#1}}
\newcommand{\WarningTok}[1]{\textcolor[rgb]{0.56,0.35,0.01}{\textbf{\textit{#1}}}}
\usepackage{graphicx}
\makeatletter
\def\maxwidth{\ifdim\Gin@nat@width>\linewidth\linewidth\else\Gin@nat@width\fi}
\def\maxheight{\ifdim\Gin@nat@height>\textheight\textheight\else\Gin@nat@height\fi}
\makeatother
% Scale images if necessary, so that they will not overflow the page
% margins by default, and it is still possible to overwrite the defaults
% using explicit options in \includegraphics[width, height, ...]{}
\setkeys{Gin}{width=\maxwidth,height=\maxheight,keepaspectratio}
% Set default figure placement to htbp
\makeatletter
\def\fps@figure{htbp}
\makeatother
\setlength{\emergencystretch}{3em} % prevent overfull lines
\providecommand{\tightlist}{%
  \setlength{\itemsep}{0pt}\setlength{\parskip}{0pt}}
\setcounter{secnumdepth}{-\maxdimen} % remove section numbering
\ifluatex
  \usepackage{selnolig}  % disable illegal ligatures
\fi

\title{Approximation of the Binomial probabilities in R}
\usepackage{etoolbox}
\makeatletter
\providecommand{\subtitle}[1]{% add subtitle to \maketitle
  \apptocmd{\@title}{\par {\large #1 \par}}{}{}
}
\makeatother
\subtitle{COMPUTER PRACTICE 2}
\author{Ivan Gercev}
\date{06 April 2021}

\begin{document}
\maketitle

git repository:
\url{https://github.com/foxybbb/TSI_Probability-Statistics}

\hypertarget{include-functions-for-bernoulli-formula-and-normal-and-poisson-approximation-from-learning-by-doing-materials-03.}{%
\subsection{Include functions for Bernoulli formula and Normal and
Poisson approximation from learning-by-doing materials
03.}\label{include-functions-for-bernoulli-formula-and-normal-and-poisson-approximation-from-learning-by-doing-materials-03.}}

\begin{Shaded}
\begin{Highlighting}[]
\NormalTok{bernoulliTrialCalc }\OtherTok{=} \ControlFlowTok{function}\NormalTok{(k, n, p) \{}
  \FunctionTok{return}\NormalTok{ (}\FunctionTok{dbinom}\NormalTok{(k, n, p))}
\NormalTok{\}}
\end{Highlighting}
\end{Shaded}

\begin{Shaded}
\begin{Highlighting}[]
\NormalTok{approxMoivreLaplaceLocal }\OtherTok{=} \ControlFlowTok{function}\NormalTok{(k, n, p) \{}
\NormalTok{  x }\OtherTok{=}\NormalTok{ (k }\SpecialCharTok{{-}}\NormalTok{ n }\SpecialCharTok{*}\NormalTok{ p) }\SpecialCharTok{/} \FunctionTok{sqrt}\NormalTok{(n }\SpecialCharTok{*}\NormalTok{ p }\SpecialCharTok{*}\NormalTok{ (}\DecValTok{1} \SpecialCharTok{{-}}\NormalTok{ p))}
  \CommentTok{\# Call the normal density function}
\NormalTok{  fi }\OtherTok{=} \FunctionTok{dnorm}\NormalTok{(x)}
  \CommentTok{\# Approximate calculation}
\NormalTok{  p }\OtherTok{=}\NormalTok{ fi }\SpecialCharTok{/} \FunctionTok{sqrt}\NormalTok{(n }\SpecialCharTok{*}\NormalTok{ p }\SpecialCharTok{*}\NormalTok{ (}\DecValTok{1} \SpecialCharTok{{-}}\NormalTok{ p))}
  \FunctionTok{return}\NormalTok{(p)}
\NormalTok{\}}
\end{Highlighting}
\end{Shaded}

\hypertarget{define-a-function-which-implements-proper-formula-selection-following-the-last-diagram-in-topic-1.5-lecture-slides.}{%
\subsection{Define a function, which implements proper formula
selection, following the last diagram in Topic 1.5 lecture
slides.}\label{define-a-function-which-implements-proper-formula-selection-following-the-last-diagram-in-topic-1.5-lecture-slides.}}

\begin{Shaded}
\begin{Highlighting}[]
\NormalTok{probMin }\OtherTok{=} \FloatTok{0.2}
\NormalTok{chooseAproximation }\OtherTok{=} \ControlFlowTok{function}\NormalTok{ (k, n, p)}
\NormalTok{\{}
  \ControlFlowTok{if}\NormalTok{ (n }\SpecialCharTok{\textgreater{}} \DecValTok{20} \SpecialCharTok{\&\&}\NormalTok{ ((p }\SpecialCharTok{\textless{}}\NormalTok{ probMin) }\SpecialCharTok{||} \DecValTok{1} \SpecialCharTok{{-}}\NormalTok{ p }\SpecialCharTok{\textless{}}\NormalTok{ probMin)) \{}
    \FunctionTok{return}\NormalTok{ (}\FunctionTok{approxMoivreLaplaceLocal}\NormalTok{(k, n, p))}
\NormalTok{  \}}
  \ControlFlowTok{else}\NormalTok{\{}
    \FunctionTok{return}\NormalTok{ (}\FunctionTok{bernoulliTrialCalc}\NormalTok{(k, n, p))}
\NormalTok{  \}}
\NormalTok{\}}
\end{Highlighting}
\end{Shaded}

\begin{Shaded}
\begin{Highlighting}[]
\FunctionTok{chooseAproximation}\NormalTok{(}\DecValTok{10}\NormalTok{, }\DecValTok{25}\NormalTok{, }\FloatTok{0.5}\NormalTok{)}
\end{Highlighting}
\end{Shaded}

\begin{verbatim}
## [1] 0.09741664
\end{verbatim}

\hypertarget{use-the-developed-function-to-solve-all-problems-from-practice-3-and-practice-4-include-the-problem-statementsinto-the-markdown}{%
\subsection{Use the developed function to solve all problems from
Practice 3 and Practice 4 (include the problem statementsinto the
markdown)}\label{use-the-developed-function-to-solve-all-problems-from-practice-3-and-practice-4-include-the-problem-statementsinto-the-markdown}}

\hypertarget{practise-3}{%
\subsubsection{Practise 3}\label{practise-3}}

\begin{Shaded}
\begin{Highlighting}[]
\NormalTok{p }\OtherTok{=} \FloatTok{0.5}
\NormalTok{n }\OtherTok{=} \DecValTok{6}
\NormalTok{k }\OtherTok{=} \DecValTok{4}
\NormalTok{pr }\OtherTok{\textless{}{-}} \FunctionTok{bernoulliTrialCalc}\NormalTok{(k,n,p)}
\FunctionTok{print}\NormalTok{(pr, }\AttributeTok{useSource =} \ConstantTok{TRUE}\NormalTok{)}
\end{Highlighting}
\end{Shaded}

\begin{verbatim}
## [1] 0.234375
\end{verbatim}

\hypertarget{practise-4}{%
\subsubsection{Practise 4}\label{practise-4}}

\hypertarget{implement-the-cumulative-distribution-functions}{%
\subsection{Implement the cumulative distribution
functions}\label{implement-the-cumulative-distribution-functions}}

\[
  \begin{equation*}
    F_{binom}(k,n,p)= P\{K \leq k\} = \sum_{i=0}^k C_n^i p^i(1-p)^{n-i} 
  \end{equation*}     
\]

\begin{Shaded}
\begin{Highlighting}[]
\CommentTok{\#comulativeBinomalDistribCalc = function(k,n,p)\{}

  
\CommentTok{\#\}}
\end{Highlighting}
\end{Shaded}

\$\$ \begin{equation*}
    F_{normal}(k,n,p)= \phi \left( \frac{k-np}{\sqrt{np(1-p)}} \right)
  \end{equation*}

\$\$

\begin{Shaded}
\begin{Highlighting}[]
\CommentTok{\#comulativeNormalDistribCalc = function(k,n,p)\{}
 \CommentTok{\# pnorm()}
\CommentTok{\#\}}
\end{Highlighting}
\end{Shaded}

\hypertarget{implement-a-function-for-calculating-the-approximation-accuracy-metric-as}{%
\subsection{Implement a function for calculating the approximation
accuracy metric
as}\label{implement-a-function-for-calculating-the-approximation-accuracy-metric-as}}

\[
  D(n,p) = max|F_{normal}(k,n,p) - F_{binom}(k,n,p)| 
\]

\end{document}
